\documentclass[sigconf]{acmart}

\usepackage{booktabs} % For formal tables


% Copyright
%\setcopyright{none}
%\setcopyright{acmcopyright}
%\setcopyright{acmlicensed}
\setcopyright{rightsretained}
%\setcopyright{usgov}
%\setcopyright{usgovmixed}
%\setcopyright{cagov}
%\setcopyright{cagovmixed}


% DOI
%\acmDOI{10.475/123_4}

% ISBN
%\acmISBN{123-4567-24-567/08/06}

%Conference
\acmConference[CNW]{SJTU Computer Network Workshop}{December 2017}{Shanghai, China} 
\acmYear{2017}
\copyrightyear{2017}


%\acmArticle{4}
%\acmPrice{15.00}

% These commands are optional
%\acmBooktitle{Transactions of the ACM Woodstock conference}
%\editor{Jennifer B. Sartor}
%\editor{Theo D'Hondt}
%\editor{Wolfgang De Meuter}


\begin{document}
\title{Predicting Internet Path Dynamics and Performance with Machine Leanring}
%\titlenote{Produces the permission block, and
%  copyright information}
%\subtitle{Extended Abstract}
%\subtitlenote{The full version of the author's guide is available as
%  \texttt{acmart.pdf} document}


\author{Zhenghui Wang}
\affiliation{%
  \institution{Shanghai Jiao Tong University}
  \city{Shanghai} 
  \state{China} 
  \postcode{200240}
}
\email{felixwzh@outlook.com}

\author{Hao Wang}
\affiliation{%
	\institution{Shanghai Jiao Tong University}
	\city{Shanghai} 
	\state{China} 
	\postcode{200240}
}
\email{?@?.com}

\author{Yuheng Zhi}
\affiliation{%
	\institution{Shanghai Jiao Tong University}
	\city{Shanghai} 
	\state{China} 
	\postcode{200240}
}
\email{?@?.com}

\author{Shukai Liu}
\affiliation{%
	\institution{Shanghai Jiao Tong University}
	\city{Shanghai} 
	\state{China} 
	\postcode{200240}
}
\email{?@?.com}

% The default list of authors is too long for headers.
\renewcommand{\shortauthors}{CN Group}


\begin{abstract}
TODO
\end{abstract}

%
% The code below should be generated by the tool at
% http://dl.acm.org/ccs.cfm
% Please copy and paste the code instead of the example below. 
%
\begin{CCSXML}
<ccs2012>
 <concept>
  <concept_id>10010520.10010553.10010562</concept_id>
  <concept_desc>Computer systems organization~Embedded systems</concept_desc>
  <concept_significance>500</concept_significance>
 </concept>
 <concept>
  <concept_id>10010520.10010575.10010755</concept_id>
  <concept_desc>Computer systems organization~Redundancy</concept_desc>
  <concept_significance>300</concept_significance>
 </concept>
 <concept>
  <concept_id>10010520.10010553.10010554</concept_id>
  <concept_desc>Computer systems organization~Robotics</concept_desc>
  <concept_significance>100</concept_significance>
 </concept>
 <concept>
  <concept_id>10003033.10003083.10003095</concept_id>
  <concept_desc>Networks~Network reliability</concept_desc>
  <concept_significance>100</concept_significance>
 </concept>
</ccs2012>  
\end{CCSXML}

%\ccsdesc[500]{Computer systems organization~Embedded systems}
%\ccsdesc[300]{Computer systems organization~Redundancy}
%\ccsdesc{Computer systems organization~Robotics}
%\ccsdesc[100]{Networks~Network reliability}


\keywords{TODO}


\maketitle

\section{Introduction}

TODO

\section{Related Works}
TODO
\subsection{Type Changes and {\itshape Special} Characters}
You can indicate italicized words or phrases in your text with the command \texttt{{\char'134}textit}; emboldening with the command \texttt{{\char'134}textbf} and typewriter-style (for instance, for computer code) with \texttt{{\char'134}texttt}.  
\subsection{Math Equations}
You may want to display math equations in three distinct styles:
inline, numbered or non-numbered display.  Each of
the three are discussed in the next sections.

\section{Methodology}
TODO
\section{Experiment}
TODO
\section{Conclusions}
TODO

\begin{acks}
TODO
\end{acks}



\bibliographystyle{ACM-Reference-Format}
\bibliography{bibliography} 

\end{document}
